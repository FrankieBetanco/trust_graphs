\documentclass{article}
\usepackage[utf8]{inputenc}

\title{CS5413 Programming Assignment 2}
\author{Frankie Betancourt }
\date{\today}

\begin{document}

\maketitle

\section{Introduction}
This programming assignment focuses on examining a weighted trust graph created
from privacy and anonymity graphs. The implementation is in C++, and was done 
in a subclass that extends the vanilla PPAI graph from PA1. Then, once the graph was
created, code was written to identify strongly-trusted components, 
shortest-trust-path, and minimum spanning trust tree. Additionally, a new trust 
variable beyond PPAI was created called ``interaction'', that enhances how trust
is computed. 

\section{Overall data structure description} 
The main data structure is an adjacency matrix that contains the weighted trust edges.
The computation of the edge weights will be discussed in greater detail later. 

\section{Computation of Trust} 
There are two types of edges in the graph: privacy edges, and anonymity edges. The
interpretation is that a directed edge from one user to another indicates that the 
attribute in question is to be kept either anonymous, or private from the terminating
end of the edge. This results in edges increasing the ``distrust'' of a user, and thus
a higher indegree of a node indicates more distrust, and a higher outdegree indicates
that a user distrusts more people in general. 

Recall that the users permissions are represented as a number. Viewing the number in
binary, the \(n^{th}\) bit in the number represents the \(n^{th}\) user attribute. A
value of 1 for the bit indicates that they want to have an edge of the type of   
that attribute beginning at the current user, and terminating at all other      
users. To start, the anonymity and privacy permissions are the bit-wise complement of 
one another. This results in initially having all users fully distrusting one another.

I chose to give each attribute a weight, with all of the weights adding up to 1. For 
example, the weight of name is 25\% and age is 25\%, so if user 1 has name and age
(anonymity or privacy) edges to user 2, then user 1 only trusts user 2 50\%. The other
factor taken in to account is the user interaction score. 

\section{User Interaction}
For the new variable beyond PPAI, I chose to include user interaction. A positive 
interaction score indicates that a user mostly trusts other users, and negative one
indications that the user mostly makes changes that distrust other users. Initially, 
all users interaction starts at 0, and can range between -1 and 1. The computation is
as follows: if a user makes a change to add an edge, then the interaction score is
then decreased by 0.01, and if the user chooses to remove an edge, then the 
interaction score is increased by 0.01. The overall interaction score is also 
taken into account when computing node-to-node trust. This lets the trust and distrust
computation have some amount of ``momentum'' incoporated in to it: users that tend to
distrust other users will contribute less strongly to the overall trustedness of a 
user, and vice versa. 

So, the computation of trust is the sum of the weights of all attribute edges, 
as well as the overall trust score. For example, if the sum of all the edges
weights from user 1 to user 2 is .5, and the overall user interaction score is -.1, then the weight of the edge from user 1 to user 2 is .4.

\section{Program: trust\_graph}
For convenience, test inputs are provided, with 10, 50, and 100 users. They can be 
found in the directory src/trust\_graph/test\_inputs. 

\section{Building}                                                              
Submitted is a tarfile containing the repository. After untarring, simply       
change directory to "src", and run "make". This should make all of the executables.   

\subsection{Usage}
Usage is as follows: 
\begin{verbatim}
trust_graph [input_file] [n_events] [node/pair/all/print]
\end{verbatim}
The input file is a csv in the format as outlined in PA1, and examples are provided 
in the directory src/trust\_graph/test\_inputs. The n\_events parameter specifies the 
number of ``trust\_events'' that should be performed. What this means is that for
each event, a privacy or anonymity edge is added or removed from the graph. The 
probability of adding an edge is 50\%, and the same for removing. The probability of
the edge type being privacy 50\%, and the same for anonymity. Then, the third 
argument lets you mak queries about a single node, a pair of nodes, all nodes, 
or print out the trust\_graph. 

\subsubsection{Node operations}
For the ``node'' type operations, there are 3 types of queries that can be made: 
indegreee, outdegree, and trustworthiness. 

An example of an indegree query is: 
\begin{verbatim}
./trust_graph 10_people.csv 10000 node 10 indegree
\end{verbatim}
This command will read create a trust graph for the input file ``10\_people.csv'', 
do 10000 trust events, and then query the indegree of node 10. In my case, the output
will be: 
\begin{verbatim}
    attribute      privacy        anonymity      
---------------------------------------------
0              4              4              
1              4              5              
2              2              7              
3              2              7              
4              3              6              
5              4              5              
6              4              5              
7              0              1              
8              1              8        
\end{verbatim}

Outdegree queries are similar, only the 5th argument is changed to ``outdegree''. 

The other type of single node query that can be made is overall trustworthiness. The
way this is computed is sum up all of the weights of the incoming edges, and divide
by the number of edges in the graph. This will show how much the node is trusted 
on average. For example the command: 
\begin{verbatim}
./trust_graph 10_people.csv 10000 node 10 trustworthiness
\end{verbatim}
will compute the overall trustworthiness of node 10 after 10000 trust events: 
\begin{verbatim}
Node 10's overall trustworthiness is 0.371111
\end{verbatim}
\subsubsection{Node Pairs}
For node ``pair'' operations, one can specify two nodes, and then get either
the shortest trust path, or amount of node 1 trusts node 2. 

For example, the command: 
\begin{verbatim}
./trust_graph 10_people.csv 10000 pair 2 5 path
\end{verbatim}
will compute the shortest trust path from 2 to 5 using Dijkstra's Algorithm: 
\begin{verbatim}
Path from destination to source
6<-2<-5
Path length = 0.22
\end{verbatim}

For trustedness, the command: 
\begin{verbatim}
./trust_graph 10_people.csv 10000 pair 2 5 trustedness
\end{verbatim}
Will compute how much node 2 trusts node 5: 
\begin{verbatim}
Node1 trusts Node2: 0.69/1.0
Node2 trusts Node1: 0.22/1.0
Friendship Strength: 0.455
\end{verbatim}
The freindship strength is calculated as 
\[ \frac{\textnormal{node 1 to node 2 edge weight} + \textnormal{node 2 to node 1 edge weight}}{2}\]

\subsubsection{All}
For ``all'' type computations, this will give information about the graph as a whole. 
This will allow queries of the trustworthiness for all nodes, friendship for all nodes, strongly connected components, and minimum spanning tree. 

For example, the command 
\begin{verbatim}
./trust_graph 10_people.csv 10000 all friendship
\end{verbatim}
Will compute the friendship strength for all pairs and output a sorted list: 
\begin{verbatim}
8 to 3 friendship strength: 0.725
8 to 2 friendship strength: 0.72
8 to 4 friendship strength: 0.68
6 to 4 friendship strength: 0.63
7 to 5 friendship strength: 0.575
[output truncated]
\end{verbatim}

Trustworthiness is also similar, providing a list of all nodes sorted by the
overall trustworthiness as discussed previously. 

The command: 
\begin{verbatim}
./trust_graph 10_people.csv 10000 all components
\end{verbatim}
will provide a list of all strongly connected components using Kosaraju's Algorithm: 
\begin{verbatim}
component 0 consists of: 0 1 7 
component 1 consists of: 2 3 4 5 8 9 
component 2 consists of: 6
\end{verbatim}

The command:
\begin{verbatim}
./trust_graph 10_people.csv 10000 all mst
\end{verbatim}
will provide a list of edges that from the minimum spanning tree of the trust graph using Kruskal's Algorithm:
\begin{verbatim}
MST contains: 
0->5: 0.72
1->6: 0.71
2->7: 0.78
2->8: 0.86
[output truncated]
\end{verbatim}

Finally, the command 
\begin{verbatim}
./trust_graph 10_people.csv 10000 print
\end{verbatim}
Will print the adjacency matrix for the graph: 
\begin{verbatim}
      0      1      2      3      4      5      6      7      8      9      
-----------------------------------------------------------------------------
0   | 0      0.34   0.42   0.06   0.34   0.72   0.53   0.38   0.1    0.21   
1   | 0.13   0      0.01   0.32   0.49   0.09   0.71   0.03   0      0.34   
2   | 0      0.28   0      0.39   0.29   0.69   0.15   0.78   0.86   0.2    
3   | 0      0.33   0.32   0      0.34   0.13   0.43   0.14   0.93   0.13   
4   | 0.22   0.35   0.28   0.13   0      0.7    0.84   0.6    0.75   0.22   
5   | 0.34   0.38   0.22   0.55   0.07   0      0.3    0.94   0.08   0      
6   | 0.07   0.18   0.44   0      0.42   0.07   0      0.73   0.8    0.19   
7   | 0.53   0.53   0.28   0      0.07   0.21   0.36   0      0.39   0.51   
8   | 0.23   0.29   0.58   0.52   0.61   0.07   0.13   0.16   0      0.04   
9   | 0.15   0.3    0.31   0.78   0.27   0.33   0.45   0.4    0.38   0      
\end{verbatim}

\section{Trust History}
For this system, the trust history of an individual user cannot be kept private. The
reason is that a user is able to compute the their own trust by looking at other 
users, and seeing what information is available. If the information is not available, 
they can tell that there is an edge from the user in question to the querying user. 
This results in the user being able to tell if a user trusts or distrusts them. 
However, this only applies for a single user to all other users; they cannot see
the trust values between two arbitrary users. Thus, the trust history cannot be
guaranteed to be kept private. 

\section{Conclusion}
The proposed, trust based, semi-sovereign identity management system provides ways to
manage the various traits and their visibility to others on the system, and provides
ways to analyze graph data, and simulate user interactions. The system takes in to 
account not only the individual interactions, but the history of a users interactions
to determine the amount of weight to give to another users trust or distrust. 
\end{document}
